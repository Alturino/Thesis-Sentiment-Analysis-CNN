\pagestyle{fancy}
\fancypagestyle{plain}{}
\cfoot{\Roman{chapter}-\arabic{page}}
\rhead{}
\setcounter{page}{1}
\chapter{KESIMPULAN DAN SARAN}

\section{Kesimpulan}
Kesimpulan yang dapat diambil sebagai hasil dari penelitian ini adalah sebagai berikut:
\begin{enumerate}
  \item Perangkat lunak analisis sentimen pada opini publik tentang angkutan online dengan menggunakan
        metode CNN berbasis \emph{mobile} dapat dikembangkan dengan baik.
  \item Analisis sentimen dengan menggunakan metode CNN dapat menghasilkan nilai akurasi, precision,
        recall, dan f1-score berturut-turut 91,25\%, 91,31\%, 91,25\%, dan 91,25\%. pada data uji dengan pembagian
        data latih, data validasi, dan data uji dengan rasio 60:20:20 dan konfigurasi dengan nilai parameter
        epoch = 125, batch\_size = 8, word2vec\_trainable = false, optimizer = adam, learning\_rate = 0.00001,
        loss = mse, fully\_connected\_activation = sigmoid dan convolution\_activation = relu.
\end{enumerate}

\section{Saran}
Saran yang dapat digunakan dari penulis untuk penelitian lebih lanjut sebagai berikut:
\begin{enumerate}
  \item Menambahkan dataset karena dataset \emph{review} dengan jumlah 6000 tergolong sedikit untuk melatih CNN\@.
  \item Menambahkan metode untuk menormalisasi kata menjadi kata baku. 
  \item Mengubah emoji menjadi kata. 
  \item Mengubah angka menjadi kata.
\end{enumerate}