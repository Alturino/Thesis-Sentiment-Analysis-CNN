\pagestyle{fancy}
\fancypagestyle{plain}{}
\rhead{}
\cfoot{\Roman{chapter}-\arabic{page}}
\lhead{}
\setcounter{page}{1}
\pagenumbering{arabic}
\renewcommand\thepage{\Roman{chapter}-\arabic{page}}
\chapter{PENDAHULUAN}

\section{Pendahuluan}
Pada bab pendahuluan ini akan membahas mengenai latar belakang masalah, tujuan, manfaat
penelitian, rumusan masalah, batasan masalah, sistematika penulisan dan gambaran umum dari
keseluruhan dari kegiatan dalam penelitian.

\section{Latar Belakang}
Analisis sentimen adalah proses menganalisis pendapat dan perasaan seseorang tentang individu, entitas,
produk, isu, atau topik~\citep{Almaghrabi2020}. Analisis sentimen sudah menjadi topik studi yang aktif 
dalam pemrosesan bahasa alami, \emph{data mining} dan \emph{information retrieval}~\citep{Alshuwaier2022}. 
Analisis sentimen merepresentasikan dan memonitor secara komprehensif dari informasi yang terkait pada 
pandangan publik dan tanggapan pelanggan diberbagai entitas dan atribut yang berbeda, seperti 
media sosial, bisnis, dan berbagai produk atau pelayanan yang lain. Analisis sentimen mengklasifikasikan
opini, emosi, dan sikap terhadap suatu entitas, sentimen tersebut dapat dikategorikan sebagai kelas
atau label seperti negatif, positif, dan netral~\citep{Alshuwaier2022}. Berdasarkan pengkategorian 
kelas-kelas sentimen tersebut maka diperlukan klasifikasi untuk menentukan kelas dari sentimen-sentimen
yang ada. Terdapat beberapa pendekatan untuk melakukan analisis sentimen, seperti \emph{deep learning}.
Salah satu pendekatan \emph{deep learning} untuk klasifikasi adalah \emph{Convolutional Neural Network} 
atau CNN\@.

CNN adalah \emph{feedforward neural network} yang umumnya digunakan pada bidang \emph{computer vision}.
Pendekatan CNN terinspirasi dari bagaimana cara kerja otak manusia, dan hewan dalam mengenali
gambar~\citep{Zhang2018}. \newpage 

\fancypagestyle{plain}{}
\rhead{\Roman{chapter}-\arabic{page}}
\cfoot{}

Karena hasil yang baik pada CNN dalam pengenalan gambar, dan banyak
digunakan sebagai metode klasifikasi dibarengi dengan suksesnya klasifikasi ImageNet dengan
ConvNets dengan landasan tersebut CNN juga dapat diaplikasikan pada bidang analisis sentimen.
Untuk menerapkan CNN pada bidang analisis sentimen maka diperlukan penyesuaian data yang diberikan
pada CNN agar dapat memahami data selain gambar dengan baik. Agar CNN dapat memahami data selain 
gambar dengan baik maka teks perlu diubah menjadi vektor terlebih dahulu proses mengubah teks
menjadi vektor umumnya disebut \emph{text vectorization}.

Salah satu pendekatan untuk merepresentasikan teks menjadi vektor adalah \emph{Word Embedding}.
\emph{Word Embedding} adalah cara merepresentasikan kata menjadi vektor, setiap kata dipetakan ke 
vektor yang berisi nilai riil. Dengan mengubah kata menjadi vektor, kata yang memiliki makna yang 
sama akan direpresentasikan berdekatan~\citep{Khattak2019}. Metode yang umumnya digunakan untuk
membuat \emph{Word Embedding} adalah \emph{Word2vec}. \emph{Word2vec} adalah metode berbasis prediksi 
untuk membuat \emph{Word Embedding} dari \emph{corpus} teks secara efisien.

Penelitian pada bidang analisis sentimen menggunakan CNN dan \emph{Word2vec} sudah banyak dilakukan pada 
berbagai masalah yang berbeda. \citet{Sharma2020} menggunakan CNN dan \emph{pre-trained Word2vec}
untuk analisis sentimen pada dataset IMDb movie \emph{review} yang berjumlah 5331 berlabel positif dan
5331 berlabel negatif dengan akurasi sebesar 99,07\% pada data latih dan 82,19\% pada data uji. Dari
penelitian ini dapat disimpulkan bahwa CNN dan \emph{Word2vec} sangat efektif dan efisien untuk 
analisis sentimen.

\citet{Smetanin2019} menggunakan CNN dan \emph{Word2vec} untuk analisis sentimen pada produk \emph{review} dalam
bahasa Rusia dari kategori `Women's Clothes and Accessories' yang dilabel secara otomatis berdasarkan skor dari \emph{review}
dengan \emph{Precision} 74,71\%, \emph{Recall} 74,54\% dan \emph{F1-score} 74,31\%. Dari penelitian ini dapat diambil kesimpulan
bahwa CNN dan \emph{Word2vec} dapat melebihi kinerja dari \emph{Multinomial Naive Bayes}.

\citet{Dong2020} menggunakan CNN untuk analisis sentimen perkalimat pada dataset saham yang sudah
di praproses dengan akurasi 98,1\% dibandingkan dengan \emph{SVM (Support Vector Machine)} yang 
memiliki akurasi 90\% dan \emph{Random Forest} memiliki akurasi 93\%. Dari hasil penelitian tersebut
dapat diambil kesimpulan bahwa CNN melebihi akurasi dari \emph{SVM (Support Vector Machine)} dan 
\emph{Random Forest}.

Angkutan online merupakan salah satu solusi bagi masyarakat karena memberikan kemudahan pada
masyarakat untuk mencari penyedia jasa angkutan dan memberikan tarif yang tetap untuk penggunaan
jasa angkutan online melalui aplikasi. Namun dibalik manfaat atas solusi yang diberikan jasa
angkutan online, jasa angkutan online juga memiliki kekurangan. Dari permasalahan tersebut maka
diperlukan analisis seberapa puas sentimen masyarakat dengan aplikasi angkutan online.

Berdasarkan uraian diatas, untuk menguji kinerja dari metode CNN pada analisis sentimen maka akan 
dilakukan analisis sentimen pada dataset angkutan online.

\section{Rumusan Masalah}
Rumusan masalah pada penelitian ini ialah bagaimana mengimplementasikan metode
\emph{Convolutional Neural Network} (CNN) pada sistem analisis sentimen.
Berdasarkan masalah tersebut diuraikan juga pertanyaan penelitian sebagai berikut.

\begin{enumerate}
\item Bagaimana mengimplementasikan metode \emph{Convolutional Neural Network} (CNN) ke dalam analisis sentimen?
\item Bagaimana kinerja analisis sentimen dengan metode \emph{Convolutional Neural Network} berdasarkan pengukuran dari nilai akurasi, \emph{Precision}, \emph{Sensitivity (Recall)}, \emph{F-Measure (F1-Score)}, dan \emph{Accuracy}?
\end{enumerate}


\section{Tujuan Penelitian}
Tujuan penelitian ini adalah:

\begin{enumerate}
\item Menghasilkan perangkat lunak berbasis \emph{mobile} analisis sentimen meggunakan metode \emph{Convolutional Neural Network}.
\item Mengetahui nilai hasil prediksi analisis sentimen \emph{Precision}, \emph{Sensitivity (Recall)}, \emph{F-Measure (F1-Score)}, dan \emph{Accuracy} dengan metode \emph{Convolutional Neural Network}.
\end{enumerate}

\section{Manfaat Penelitian}
Berikut merupakan manfaat dari penelitian ini:

\begin{enumerate}
    \item Mengetahui sentimen masyarakat terhadap topik Angkutan Online.
    \item Hasil penelitian dapat digunakan sebagai rujukan penelitian di bidang analisis sentimen.
\end{enumerate}

\section{Batasan Masalah}
Batasan masalah dalam penelitian ini:

\begin{enumerate}
    \item Data yang digunakan berupa komentar atau kicauan.
    \item Data yang digunakan tidak mengandung emotikon dan angka.
    \item Data yang digunakan adalah teks dalam Bahasa Indonesia.
    \item Data latih dan test yang digunakan berupa file \emph{.csv}.
    \item Data diambil dari platform aplikasi \emph{Google Play Review}.
    \item Penelitian ini membagi sentimen menjadi 2 kelas atau label yaitu positif dan negatif.
\end{enumerate}

\section{Sistematika Penulisan}
Penulis dalam menulis laporan penelitian ini mengikuti sistematika penulisan pada skripsi
Fakultas Ilmu Komputer Universitas Sriwijaya. Berikut merupakan Sistematika yang diterapkan
penulis:

\subsection{BAB I PENDAHULUAN}
Bab ini akan membahas mengenai latar belakang masalah, perumusan masalah, tujuan,
manfaat penelitian, rumusan masalah, batasan masalah, sistematika penulisan dan

\subsection{BAB II KAJIAN LITERATUR}
Bab ini akan membahas landasan teori yang digunakan dalam penelitian tugas akhir ini,
seperti definisi-definisi analisis sentimen, praproses, tokenisasi, word embedding, \emph{Word2vec},
\emph{Convolutional Neural Network} (CNN) dan beberapa literatur mengenai penelitian lain
yang relevan dengan penelitian ini.

\subsection{BAB III METODOLOGI PENELITIAN}
Bab ini akan membahas mengenai tahapan yang akan dilakukan pada penelitian tugas akhir.
Seperti pengumpulan data, analisis data dan perancangan pembangunan system.
Setiap tahapan penelitian dijelaskan dengan kerangka kerja dan manajemen proyek penelitian
secara rinci.

\subsection{BAB IV PENGEMBANGAN PERANGKAT LUNAK}
Bab ini akan membahas tahapan yang dilakukan dalam proses pengembangan perangkat lunak
analisis sentimen menggunakan \emph{Convolutional Neural Network} (CNN) berbasis \emph{mobile}.

\subsection{BAB V HASIL DAN ANALISIS PERANGKAT LUNAK}
Bab ini akan membahas mengenai hasil dari pengembangan perankat lunak. Hasil pengujian
dan analisanya akan dijadikan dasar dari kesimpulan yang diambil dalam penelitian ini.

\subsection{BAB VI KESIMPULAN DAN SARAN}
Bab ini akan membahas mengenai kesimpulan dari semua uraian yang sudah dijelaskan
pada bab sebelumnya serta saran yang diuraikan dari hasil penelitian.

\section{Kesimpulan}
Bab ini telah membahas mengenai latar belakang pada penelitian ini seperti rumusan masalah,
tujuan penelitian, manfaat penelitian, batasan masalah dan sistematika penulisan.
Pada penelitian ini akan dilakukan penerapan dan analisa akurasi pada aplikasi analisis
sentimen berbasis \emph{mobile} menggunakan metode \emph{Convolutional Neural Network} (CNN)